\documentclass[11pt, a4papper]{article}
\usepackage[left=1in, right=1in, top=1in, bottom=1in]{geometry}
\usepackage{fancyhdr, changepage}
\pagestyle{fancy}
\fancyhf{}
\renewcommand{\headrulewidth}{0pt}
\fancyfoot[R]{\thepage}

\setlength{\parindent}{4em}

\begin{document}
\noindent\textbf{Author: } Aaron Valoroso \\
\textbf{Date: } May 25th, 2018 \\
\textbf{Topic: } Setting Up Static IP Addresses \\
\textbf{OS: } Ubuntu \\
\textbf{OS-Version: } 16.10 \\[1cm]

\textbf{Overview / Setup: } \\
\begin{adjustwidth}{3cm}{} 
To be able to follow along you either need two virtual or physical machines that are connected together. This can be done with lots of computers all connected together but we will just focus on just two machines. My first machine will be called Clone1 and the second will be Clone2 and they both will be virtual machines on VirtualBox. If you don?t know how to setup a virtual machine then look for my document on how to setup a virtual machine for your specific needs. \\
\end{adjustwidth}
%--------------------------------------------------------------------------------------------------
\indent \indent \textbf{Step-1: } \\
\begin{adjustwidth}{3cm}{} 
- Type: \textbf{su} \\
\begin{adjustwidth}{1cm}{} 
\indent - If these are your machines then you should have root or sudo privileges, but if not then this how to will not work for you. If they are your machines but you still do not have them, then Google how to gain them or edit the correct files in order to get them. \\
\end{adjustwidth}
%--------------------------------------------------------------------------------------------------
- Type: \textbf{ifconfig} \\
\begin{adjustwidth}{1cm}{} 
\indent - Here you will see the network devices for your system. They will range from \texttt{etho\#} to \texttt{enp0s\#}, and beyond. I'm using a virtual machine so mine is enp0s8. Usually physical machines use etho and virtual machines use enp0s. \\
\end{adjustwidth}
%--------------------------------------------------------------------------------------------------
-Type: \textbf{sudo nano /etc/network/interfaces} \\
\begin{adjustwidth}{1cm}{} 
\indent - You will see the auto lo device in the file, you will want to enter the following underneath the last line. Since I am using a virtual machine, I will be using my enp0s8 network device, you will need to figure out which network device you have associates to either your network or other virtual machines.\\
\end{adjustwidth}
%--------------------------------------------------------------------------------------------------
\indent \indent auto enp0s8 \\
\indent iface enp0s8 inet static \\
\indent \indent address 192.168.100.100 - ( This is an example ) \\
\indent \indent netmask 255.255.255.0    - ( This is an example ) \\

\pagebreak

%--------------------------------------------------------------------------------------------------
\noindent - Type: \textbf{sudo service networking restart} \\
\begin{adjustwidth}{1cm}{}
\indent - The example above we can do for clone1, then all you have to do is repeat the same process on clone2 but give it a different IP address. So, for clone2 just give it the following address: 192.168.100.101 \\ \\
 - After the previous step has been finished, go ahead and do the following command: \textbf{ping 192.168.100.101} from clone1 and vice versa from clone2 but with the 100 address. You should see data being sent across to either machine. \\ \\
 - Make sure to not give an address last three numbers bigger than 299, you will run into complications.
\end{adjustwidth}
%--------------------------------------------------------------------------------------------------
\end{adjustwidth}
\end{document}
