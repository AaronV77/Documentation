\documentclass[11pt, a4papper]{article}
\usepackage[left=1in, right=1in, top=1in, bottom=1in]{geometry}
\usepackage{fancyhdr, changepage}
\pagestyle{fancy}
\fancyhf{}
\renewcommand{\headrulewidth}{0pt}
\fancyfoot[R]{\thepage}

\setlength{\parindent}{4em}


\begin{document}
\noindent\textbf{Author: } Aaron Valoroso \\
\textbf{Date: } May 25th, 2018 \\
\textbf{Topic: } Using Valgrind w/ C++ \\
\textbf{OS: } Ubuntu \\
\textbf{OS-Version: } 16.04 \\[1cm]

\textbf{Overview: } \\
\begin{adjustwidth}{3cm}{} 
This document covers the basics for using Valgrind with C++ to discover memory leaks and other memory related issues. Even though that this document is pretty short with information, more will be added in the future. If you feel like I left out the most important parts or anything else please feel free to updated the document and create a new PR for it. Thanks. \\
\end{adjustwidth}

\textbf{Step-1: } Installing Valgrind on Ubuntu \\
\begin{adjustwidth}{3cm}{} 
- To install Valgrind onto Ubuntu then do the following command:  \\
\noindent \textbf{sudo apt-get install Valgrind} \\
\end{adjustwidth}

\textbf{Step-2: } Compile your C++ program \\
\begin{adjustwidth}{3cm}{} 
- Do the basic compile for your C++ code such as: \textbf{c++ -g \texttt{my\_example.cpp}} \\
\end{adjustwidth}

\textbf{Step-3: } Run the program with Valgrind \\
\begin{adjustwidth}{3cm}{} 
- Next we have to run our program with Valgrind for memory checking. Do the following: \textbf{valgrind --leak-check=yes ./a.out}
\end{adjustwidth}

\end{document}

