\documentclass[11pt, a4papper]{article}
\usepackage[left=1in, right=1in, top=1in, bottom=1in]{geometry}
\usepackage{fancyhdr, changepage}
\pagestyle{fancy}
\fancyhf{}
\renewcommand{\headrulewidth}{0pt}
\fancyfoot[R]{\thepage}

\setlength{\parindent}{4em}


\begin{document}
\noindent\textbf{Author: } Aaron Valoroso \\
\textbf{Date: } May 25th, 2018 \\
\textbf{Topic: } GitHub Cheat Sheet \\
\textbf{OS: } Ubuntu \\
\textbf{OS-Version: } 16.04 \\[1cm]

\textbf{Overview: } \\
\begin{adjustwidth}{3cm}{} 
This document covers the basics for using GitHub commands on terminal. Even though that this document is pretty short with information, more will be added in the future. If you feel like I left out the most important parts or anything else please feel free to updated the document and create a new PR for it. Thanks. \\
\end{adjustwidth}
%--------------------------------------------------------------------------------------------------
\indent \indent \textbf{Setting up GitHub on your system:} \\
\begin{adjustwidth}{3cm}{} 

\textbf{git config --global user.name "put your user name here"}
\begin{adjustwidth}{1cm}{} 
- This command will setup your username, so that every time you commit or push GitHub ill understand who is doing the pushing and committing.
\end{adjustwidth}

\noindent \textbf{git config --global user.email "put your email here"}
\begin{adjustwidth}{1cm}{} 
- This command will setup your email, so that every time you commit or push GitHub ill understand who is doing the pushing and committing.
\end{adjustwidth}

\noindent \textbf{git init repo-name}
\begin{adjustwidth}{1cm}{} 
- This command will create a new git repository.
\end{adjustwidth}

\noindent \textbf{git clone "URL"}
\begin{adjustwidth}{1cm}{} 
- This command will clone a repository from GitHub for you. \\
\end{adjustwidth}

\end{adjustwidth}
%--------------------------------------------------------------------------------------------------
\indent \indent \textbf{First commi to GitHut:} \\
\begin{adjustwidth}{3cm}{}

\noindent \textbf{git add}
\begin{adjustwidth}{1cm}{} 
- This command requires another argument after add. Two examples that I can give is either '.', which will add everything in the repo, or an absolute path to the file that you want to add to a commit.
\end{adjustwidth}

\noindent \textbf{git commit -m "some message"}
\begin{adjustwidth}{1cm}{} 
- This command will create a new commit that will house all of your current changes and associate a message to your commit.
\end{adjustwidth}

\noindent \textbf{git remote add origin "URL"}
\begin{adjustwidth}{1cm}{} 
- This command requires that you have a repository created on GitHubs website. I have not seen any other way than to physically go to their website and create a new repo. When you create a new repo, you can get a URL that you can sync up your repo too.
\end{adjustwidth}

\noindent \textbf{git push}
\begin{adjustwidth}{1cm}{} 
- This command will push all your commits to the repository to GitHub.
\end{adjustwidth}

\noindent \textbf{git pull}
\begin{adjustwidth}{1cm}{} 
- This command will pull all changes from the remote down to your repository on your system.
\end{adjustwidth}

\end{adjustwidth}
%--------------------------------------------------------------------------------------------------
\pagebreak
\indent \indent \textbf{Clean-up and Utilities:} \\
\begin{adjustwidth}{3cm}{}

\noindent \textbf{git branch --delete "branch-name"}
\begin{adjustwidth}{1cm}{} 
- This command will remove a local branch on your system.
\end{adjustwidth}

\noindent \textbf{git push "remote-name" --delete "branch-name"}
\begin{adjustwidth}{1cm}{} 
- This command remove a remote branch on GitHub.
\end{adjustwidth}

\noindent \textbf{git branch}
\begin{adjustwidth}{1cm}{} 
- This command will show you all the available branches within your repository. 
\end{adjustwidth}

\noindent \textbf{git checkout -b "The Branch Name"}
\begin{adjustwidth}{1cm}{} 
- This command will create a new branch within your repository.
\end{adjustwidth}

\noindent \textbf{git checkout "The Branch Name"}
\begin{adjustwidth}{1cm}{} 
- This command will switch your git system to a different branch.
\end{adjustwidth}

\noindent \textbf{git status}
\begin{adjustwidth}{1cm}{} 
- This command will show you the status of the current branch that you are on.
\end{adjustwidth}

\noindent \texttt{\textbf{git reset --soft HEAD\^ }}
\begin{adjustwidth}{1cm}{} 
- This command revert the head of your repository back one commit. You can use the following command: \textbf{git reset HEAD path/to/unwanted/file} to remove any unwanted files within the commit.
\end{adjustwidth}

\end{adjustwidth}
%--------------------------------------------------------------------------------------------------
\end{document}










